\section{Embedded syntax considerations}
\label{sec:embed}

Having focused above on host language considerations,
we now switch focus to considerations for languages
to \emph{be embedded} as matchertext.
The languages of interest for embedding
overlaps heavily with those of interest as host languages;
we separate these discussions mainly to emphasize the orthogonality
of host- and embedded-language issues.

It is already readily feasible to write valid matchertext
in most of the languages we will consider for embedding.
This is because most popular machine-readable languages
already largely conform to the ``matchers must match'' rule
in their explicit uses of the matcher characters.
Violations of the matchertext rule most commonly occur
only in embedded ``free-form'' text such as string literals and comments.
The language extensions we will propose are motivated almost exclusively
by increasing convenience and visual clarity,
and are by no means essential.

\subsection{String literals in C-like languages}

Almost certainly the most common context in which unmatched matchers
appear in most today's existing source code is within string literals.
This is especially true of code to print, or parse,
machine-readable code in most any syntax.
Structured pretty-printing code frequently includes code sequences like this:

\begin{quote}
\verb|print("[")| \\
\emph{output all elements of a list} \\
\verb|print("]")|
\end{quote}

Similarly, parsing code often uses \verb|if|, \verb|switch|,
or \verb|case| conditionals
to recognize and parse matcher-delimited syntactic structures,
as in:

\begin{quote}
\verb|if peekNextChar() == '[':| \\
\verb|  scanChar('[')| \\
\verb|  |\emph{scan all elements of a list} \\
\verb|  scanChar(']')| 
\end{quote}

Printing and scanning code like this generally violates the matchertext rule,
and adapting such code most likely represents the biggest ``pain point''
in any venture to write readily-embeddable matchertext.

Almost all programming language already offer a workable
if slightly cumbersome solution:
simply replace unmatched matchers in string literals
with suitable numeric character escapes.
Instead of \verb|print("[")|, for example,
write \verb|print("\x5B")| (C, C++, JavaScript)
or \verb|print("\u005B")| (Java, JavaScript, Go).
This always works;
the main annoyance is that it requires the writer (and reader) of the code
to remember or look up the codes for the matcher characters in an ASCII table.

The usual solution in C-like languages
to handle ``special'' characters in string literals
is simply to backslash-escape the special character,
like \verb|\[|.
This traditional solution does not work for unmatched matchers in matchertext,
however,
because the matchertext rule is deliberately language-independent
and oblivious to language-specific syntax such as that of string literals.
So an ``escaped'' unmatched bracket \verb|\[|
is just as much a matchertext violation as the bracket alone.

There is a solution that avoids the need for ASCII tables, however.
Because literal matchers are a problem in matchertext only when unmatched,
we can simply introduce escape sequences that incorporate
\emph{both} matchers as a properly-matched pair,
while ``selecting'' only the opener or closer of the pair.
In C-like languages, for example,
we suggest the sequence \verb|\o()| to escape an open parenthesis,
\verb|\c()| to escape a close parenthesis.
Similarly,
\verb|\o[]| and \verb|\c[]| represent open/close square brackets,
and \verb|\o{}| and \verb|\c{}| represent open/close curly braces.

The choice of the letters \verb|o| and \verb|c| to escape the matchers
is consistent with their standardized character classes:
\href{https://www.compart.com/en/unicode/category/Ps}{``Open Puntuation (Ps)''}
and
\href{https://www.compart.com/en/unicode/category/Pe}{``Close Punctuation (Pe)''},
respectively.
We might consider \verb|l| and \verb|r| for ``left'' and ``right'',
escept \verb|\r| is a near-universal escape for carriage return (CR).
A few languages already use \verb|o| or \verb|c| in escape sequences:
\eg Raku uses \verb|\o[|$n$\verb|]|
to denote the ASCII character with octal value $n$,
and uses \verb|\c[|$n$\verb|]|
to denote a Unicode character with name or decimal value $n$.
Many of these existing uses are technically not in conflict syntactically,
provided the existing use requires a non-empty string between the matchers --
as Raku does in the above cases, for example.
% see raku/escapes.raku for a trivial script with which to verify this.
In any case, different languages need not agree
on specific escapes sequences for unmatched matchers
and are free to make their own stylistic choices.


\xxx{relationship: triple-quoted/multiline literals}


\subsection{Comments and derived documentation}

Another context in which unmatched matchers may regularly appear
in typical source code is within comments:
\eg as part of human-readable text \emph{describing}
how the associated code handles particular characters.
Conventional language processors usually just ignore unmatched matchers
(along with everything else) in a comment.
But the matchertext discipline operates below and oblivious to
the syntax of a particular language,
and hence does not know what a ``comment'' is --
so the matchertext discipline must disallow unmatched matchers even in comments.

Since comments are generally intended for humans reading the source code,
it is usually possible simply to rephrase the comment
to avoid a literal use of unmatched matcher characters:
\eg just name it (`open parenthesis')
instead of writing it (`\verb|(|').
Another alternative,
if a language adopts the above extensions for string literals,
is simply to use these matchertext-friendly escapes in comments as well
(\eg \verb|\o()|).

In some languages,
comments often get used to produce API documentation,
using tools like \href{https://www.oracle.com/java/technologies/javase/javadoc-tool.html}{Javadoc}
or \href{https://pkg.go.dev/golang.org/x/tools/cmd/godoc}{godoc}.
In such cases,
it may be useful to interpret escape sequences such as those above
while auto-generating documentation from source code,
so that a documentation comment like `\verb|// Parse a \o()|'
becomes `Parse a (' in the formatted output generated from the code.


\subsection{SGML-derived languages}
\label{sec:embed:ml}

Considerations similar to those above for string literals
apply when we wish to embed \ml-language markup
into other languages as matchertext.
The most common reason unmatched matchers appear in markup
is when needed in literal text being marked up:
\eg human-readable text \emph{about} the matcher characters
or syntactic constructs built from them,
or code examples that contain unmatched matchers.

As with C-style string literals,
\ml languages already offer a workaround:
simply use character references,
either named (like \verb|&lpar;|)
or numeric (\verb|&#x0028;|).
For the same reasons as above,
we may like to have extensions
offering more visually-obvious alternatives for writing matchertext:
\eg \verb|&o();| and \verb|&c();|
for open and close parentheses, 
respectively.


\subsection{Uniform resource identifiers}
\label{sec:embed:uri}

Since uniform resource identifier (URI) syntax represents
a special-purpose ``little language'' just for expressing identifiers,
a primary use of URIs is to be embedded in other contexts --
software source code, documentation markup, configuration files, etc.
Especially since URIs are intended to be human-readable,
it would thus seems useful if URIs
could be maximally ``friendly'' for embedding.

\subsubsection{The near-matchertext-compliance of URI syntax}

Conventional URI syntax~\cite{rfc3986}
already ``nearly'' complies with the ``matchers must match'' rule
and is thus, usually, embeddable verbatim in a matchertext context.
Curly braces are formally disallowed in URIs.
Square brackets are allowed \emph{only} to surround IPv6 addresses
in the authority field,
in properly-matched fashion.
Thus, the only unmatched matchers that \emph{can} exist
in a strictly-valid URI are parentheses.
Even these, when appearing in URIs,
often still come in matched pairs anyway.\footnote{\begin{tiny}
	For example:
	\texttt{https://en.wikipedia.org/wiki/URI\_(disambiguation)}
	\end{tiny}}

In the rare cases when unmatched parentheses are ``needed'' in a URI,
they may always be percent-escaped as \verb|%28| or \verb|%29|.
We could always consider escaping extensions
such as \verb|%o()| and \verb|%c()|,
but it is far from clear that their likely-marginal need
would justify the syntactic complexity in this case.
Even if URI syntax is liberalized further to allow
square brackets and/or curly braces in components,
it is unclear how commonly unmatched matchers would be needed,
since it is not particularly common to write parsing or scanning code
within a URI for example.


\subsubsection{The URI end-finding problem}

Nevertheless, 
URI syntax does suffer from at least one significant usability flaw
arising from its frequent use as an embedded syntax.
URIs can and often do appear almost ``anywhere''
in freeform human-readable text --
\eg typed or copied into E-mails, notes, documents, etc.
Smart text editors often try to detect URIs on entry
and automatically turn them into hyperlinks --
but these heuristics can easily break because
there is no unambiguous syntactic separation between the URI
and surrounding (particularly following) text.
Suppose for example that I type or copy this text into an E-mail:

\begin{footnotesize}
\begin{quote}
\verb|My site is https://bford.info/index.html.|
\end{quote}
\end{footnotesize}

The trailing period (\verb|.|) \emph{could} be part of the URI,
but in this case was probably intended to terminate my English sentence.
I could try to ``armor'' the URI, like this:

\begin{footnotesize}
\begin{quote}
\verb|See my site (https://bford.info/index.html).|
\end{quote}
\end{footnotesize}

But the close parenthesis, as well, \emph{could} be part of the URI
and be sucked into the link by a ``greedy'' URI auto-recognizer,
resulting in a broken link.
A careful reader of Appendix C of the URI specification~\cite{rfc3986}
might find the recommendation to delimit URIs
with angle brackets \verb|<>| --
but rather few people seem to be aware of this recommendation in practice,
let alone are following it.


\subsubsection{Matchertext resource identifiers}
\label{sec:embed:mri}

Given how commonly URIs are embedded in both freeform human-readable text
as well as other machine-readable syntaxes of all kinds,
we suggest that a more useful and ambitious potential evolution
would make URI syntax \emph{self-delimiting}.
In particular,
let us consider an alternative potential URI syntax
in which we surround the URI's body -- everything after the scheme name --
with square brackets instead of separating it from the body with a colon.
Thus, \verb|http://my.site/| becomes \verb|http[//my.site/]|.
This alternate syntax uses only characters
that are already used (and reserved) in current URI syntax,
and it remains readily recognizable in freeform embedded contexts,
but now the end can always be found unambiguously with no heuristic guessing.

Let's call this new syntax
a \emph{matchertext resource identifier} or MRI.
Since MRI syntax is distinct and not readily confused with traditional URIs,
it could enforce the rule that all URI body content within the brackets
must be matchertext --
\ie that unmatched matchers in the body must be percent-encoded --
for verbatim embedding of other syntaxes (or other MRIs) in the body.
Just as IRIs~\cite{rfc3987} liberalized URI syntax
while preserving backward compatibility
by defining automatic conversions in both directions,
MRI syntax could similarly be converted automatically
to or from traditional URI and IRI syntax.

Assume that MRI syntax includes the extensions
discussed earlier in \cref{sec:host:uri} --
in particular the rule that a square bracket sequence \verb|[|$m$\verb|]|
nested within the URI body protects the embedded matchertext $m$
from percent-encoding in the outer context.
With this syntax, MRIs cleanly nest with no escaping needed,
not even to introduce a matchertext embedding context.
An embedded MRI appearing in a path or query string component
of a host MRI never need be escaped, for example,
as illustrated by the examples in \cref{fig:search-query}.

Moreover, MRI syntax could potentially be \emph{simpler}
than traditional URI syntax,
because complex and rarely-used sub-syntaxes such as IPv4 and IPv6 addresses
could be ``broken out'' of the main MRI syntax
and handled instead as embedded MRIs in the host MRI's authority field.
For example,
the URI `\verb|http://1.2.3.4:80/|' would become
the MRI `\verb|http[//ip4[1.2.3.4]:80/|', and
the URI `\verb|http://[1234::abcd]:80/|' would become
the MRI `\verb|http://ip6[1234::abcd]:80/|'.
%IPv6 addresses with scoped identifiers
%could avoid obfuscation in URIs~\cite{rfc6874}:
%the URI `\verb|http://[1234::abcd%25eth0]/' becomes
%the MRI `\verb|http://ip6[1234::abcd%eth0]/'.
The MRI host field syntax knows only about domain names or nested MRIs,
not about IP addresses.


\subsubsection{How liberal to liberalize?}
\label{sec:embed:mri:liberal}

spaces, control codes.



\subsection{Regular expressions}

\xxx{
Character classes: no matchers unless both present.
But character escapes instead...
}

Regular expressions in
\href{https://www.tcl.tk/man/tcl8.4/TclCmd/re_syntax.html#M32}{Tcl}
and
\href{https://developer.mozilla.org/en-US/docs/Web/JavaScript/Guide/Regular_Expressions/Character_Classes#types}{JavaScript}
use \texttt{\\c}$X$ as an escape sequence for control code characters.
Using this syntax with an opener as the character $X$
appears to be illegal in JavaScript regular expressions,
and perhaps legal but unlikely to be used in Tcl regular expressions.

\xxx{
Question: how often are these escape sequences used,
and how often are the equivalent sequences for introducing control codes used
(\eg single-letter sequences, or octal or hex numeric sequences)?
Which specific escape sequences does this escape sequence get used for
and with which characters $X$?

Relevant: \href{https://www.tcl.tk/man/tcl8.5/tutorial/Tcl21.html}{More Quoting Hell - Regular Expressions 102}.
Also \href{https://wiki.tcl-lang.org/page/Quoting+hell}{Quoting hell}
}



\subsection{Trouble spots in syntax tradition}

\xxx{ When are unmatched matchers traditionally used
in actual language syntax, not just as literal text embedded within syntax?
Mathematical half-open range/set notation.
How problematic is this?
Other examples?
}

\subsection{Future: other oft-embedded languages}

\xxx{	Other often-embedded languages to consider, potentially:
		SQL
		JSON
}

