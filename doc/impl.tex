\section{Implementations of matchertext}
\label{sec:impl}

This section is mainly a placeholder at the present.
Some preliminary work is underway
implementing experimental matchertext extensions
for several languages and embedding-oriented syntaxes.
This section is intended to be expanded
as we gain experience
implementing and using matchertext extensions.

For those wishing to help with implementation and experimentation,
the following are some of the key work items
enabling us to start experimenting with matchertext in the context
of any particular language of interest:
\begin{itemize}
\item	A brief specification of proposed
	syntactic extensions for hosting matchertext
	and/or conveniently writing embedded matchertext,
	adapted to the specific language of interest,
	with clearly-defined rationale for the particular syntax choices.
\item	An implementation of those extensions for hosting and/or embedding,
	selectively enabled via configuration parameters
	for backward compatibility,
	in some mature processor for the language of interest
	(\eg a compiler, interpreter, or library implementation).
\item	A configurable extension to the language processor
	that optionally causes it
	to check and enforce the matchertext discipline
	in processed text:
	\ie to verify that matchers match in source files or strings.
\item	Purely for experimentation purposes,
	an extension of some processor for the language
	that can analyze ``legacy'' source files in the language
	to detect and categorize matchertext violations
	(\eg whether in string literals, comments, or elsewhere).
	We will use this to help estimate the likely ``pain''
	of adopting the matchertext discipline in the language
	and how commonly this pain would affect typical code today.
\end{itemize}


\subsection{MinML: minified matchertext markup}

One early experiment in matchertext-friendly syntax design
is MinML~\cite{ford22minml},
an alternative syntax for SGML-derived markup languages like HTML and XML.

Beyond merely adding matchertext hosting and embedding extensions
as discussed in the sections above,
MinML more ambitiously reformulates the basic \ml syntax
to rely on matching brackets for basic structure
rather than matching start/end tags as in SGML tradition.
For example, \emph{emphasis} is written like
\verb|em[emphasis]| rather than \verb|<em>emphasis</em>|.
Character references are written like
\verb|[star]| instead of \verb|&star;|.

A ``quotation'' delimited by matching quote characters
may be written like
\verb|"[quotation]| in MinML instead of \verb|&ldquo;quotation&rdquo;|.
A comment is \verb|-[comment]| instead of \verb|<!--comment-->|.
A raw embedded text sequence
is written \verb|+[verbatim]| instead of \verb|<![CDATA[verbatim]]>|.
MinML's embedded sequences leverage matchertext
to support arbitrary nesting,
so a verbatim example of a raw matchertext sequence
is simply \verb|+[+[example]]|, rather than in XML:

\begin{footnotesize}
\begin{center}
\verb|<![CDATA[<![CDATA[example]]]]><![CDATA[>]]>|
\end{center}
\end{footnotesize}

For escaping unmatched matchers,
MinML supports both the traditional HTML named and numeric character references,
and bracket-delimited versions of the ``visual'' matcher escapes
suggested earlier in \cref{tab:unmatched-matchers}
and \cref{sec:embed:re:class} discussing regular expressions:

\begin{center}
\begin{tabular}{lccc}
Matchers	&		& Open		& Close		\\
\hline
Parentheses	& \verb|()|	& \verb|[(<)]|	& \verb|[(>)]|	\\
Brackets	& \verb|[]|	& \verb|[[<]]|	& \verb|[[>]]|	\\
Braces		& \verb|{}|	& \verb|[{<}]|	& \verb|[{>}]|	\\
\end{tabular}
\end{center}

An experimental library and command-line tool
to parse MinML and convert it to HTML or XML,
written in Go,
is available at \url{https://github.com/dedis/matchertext}.
An \href{https://github.com/bford/hugo}{extention}
to the \href{https://gohugo.io}{Hugo}
static website generator supports web authoring in MinML.

