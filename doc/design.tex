\section{Matchertext design and rationale}
\label{sec:design}

This section first defines matchertext 
as an abstract mathematical syntax in \cref{sec:design:abstract},
then in \cref{sec:design:concrete}
as a concrete syntactic discipline
pragmatically inspired by predominant practices.
These definitions represent the ``core'' of the matchertext concept,
and are the \emph{only} rules
that different languages must ``agree on''
in order to achieve the goal of escapeless interlanguage embedding.

\subsection{Abstract definition of matchertext}
\label{sec:design:abstract}

We first define \emph{abstract matchertext}
in order to ensure that the basic concept is clear and precise.

We assume at the outset we are given some arbitrary alphabet $\Sigma$,
along with some finite set $\Pi$ of character pairs 
$\{(o_1,c_1),\dots,(o_k,c_k)\}$,
such that $\{o_i, c_i\} \subseteq \Sigma$ for all $1 \le i \le k$.
We define the \emph{openers} $O$ as the set of characters $o$
such that some pair $(o, c) \in \Pi$.
Similarly, the \emph{closers} $C$ are the characters $c$
such that some pair $(o, c) \in \Pi$.
We assume and require that the sets of openers and closers do not overlap:
\ie $O \cap C = \emptyset$.
We define the \emph{matchers} $M$ as the set of all openers and closers
(\ie $M = O \cup C$).
We define the \emph{nonmatchers} $N$ as
all characters that are not matchers
(\ie $N = \Sigma \setminus M$)
A \emph{matchertext configuration}
is a pair $(\Sigma, \Pi)$ following the above rules.

We now define the language $L$ of \emph{matchertext strings}
inductively (\ie generatively) as follows:
\begin{itemize}
\item	Any string $n$ consisting exclusively of nonmatcher characters in $N$,
	including the empty string,
	is a matchertext string in $L$.
\item	For any matchertext strings $m_1,m_2,m_3$ in $L$,
	and for any pair $(o,c) \in \Pi$,
	the concatenation $m_1||o||m_2||c||m_3$
	is also a matchertext string in $L$.
\end{itemize}

Intuitively, this definition captures the basic rule
that \emph{matchers must match} in matchertext.
Openers can be introduced into valid matchertext
only when paired with a matching closer, and vice versa,
but nonmatchers may be interspersed throughout with no constraints.

We consider matchertext to constitute a purely syntactic rule
with no associated semantic meaning.
While we can formally define it as a syntactic language,
in practice we will refer to it as a \emph{syntactic discipline}
rather than a language
because it assigns no specific meaning or purpose to the strings in $L$,
and no structure apart from the basic rule that matchers must match.
The meanings and structural purposes of all characters --
both matchers and nonmatchers --
are deliberately left entirely open
for any particular ``matchertext-compliant'' language to define.

\xxx{ formally prove
Always-embeddability property given that embedded string is valid matchertext.
}

\subsubsection{Escapeless embedding in matchertext}
\label{sec:design:abstract:embed}

Suppose there is a set of languages $\mathcal{L} = \{L_1,\dots,L_k\}$
whose members all agree on a common  matchertext configuration $(\Sigma,\Pi)$.
Any language $L_h \in \mathcal{L}$ can \emph{host}
embedded strings in any other language $L_e \in \mathcal{L}$
without escaping or other transformations to the strings from $L_e$,
provided $L_h$ enforces the following simple \emph{embedding rule}.
Any embedded matchertext string $m \in L_e$ 
must be delimited (surrounded) by
some pair of strings $s_o, s_c$ defined by the host language $L_h$,
such that the full embedding sequence is $s_o || m || s_c$ in $L_h$.
Further, $s_o$ must contain one or more open matchers in $O$
that would be unmatched in $s_o$ alone --
that is, $s_o$ alone \emph{cannot} be valid matchertext --
and $s_c$ must contain one or more corresponding close matchers in $C$
that would be unmatched in $s_c$ alone.

This embedding rule permits escapeless embedding
because no fixed characters or strings, including $s_o$ and $s_c$,
need be unconditionally forbidden in the embedded string $m \in L_e$,
provided that $m$ is valid matchertext.
The host language processor need not know anything
about the embedded language $L_e$ other than that $m$ is matchertext.
A host language processor that parses left-to-right from $s_o$
can ignore any embedded instances of $s_o$ and $s_c$ within $m$
because the matchers they contain must match within $m$.
The host language processor can unambiguously recognize
the closer string $s_c$ that terminates the embedding
because it contains at least one close matcher
that would be unmatched, and thus illegal, if it were part of $m$.

In order to guarantee that verbatim embedding always works reliably,
a host language not only can but \emph{must} refrain
from applying either transformations (\eg escapes)
or restrictions (\eg disallowing certain characters or sequences)
within the embedded matchertext it is hosting.
We can view embedded matchertext as analogous to a diplomatic embassy,
whose host country is required by international law to respect and protect
the ``involability'' of the embassy's ``premises''
and not enter or otherwise meddle in
the embassy's internal affairs~\cite[Article 22]{un61vienna}.
Beyond the matchertext rule that ASCII matchers must match,
an embedded language's syntactic affairs are exclusively its own,
not to be meddled in by a host language.
While exceptions to this rule may sometimes be justified
as we discuss below in \cref{sec:design:concrete:variations},
any exceptions inevitably reduce verbatim embedding compatibility.


\subsection{Concrete matchertext in practice}
\label{sec:design:concrete}

The abstract definition of matchertext above
and its basic structural rule apply in principle
to any matchertext configuration $(\Sigma,\Pi)$.
In practice, however,
we must standardize on particular choices of $\Sigma$ and $\Pi$
across a set of languages of interest
in order to achieve escapeless embedding among them.
We wish to identify a particular, concrete matchertext configuration
that fits existing syntactic syntactic practices as well as possible,
and facilitates escapeless embedding across 
minimally-adapted variants of today's popular machine-readable languages.

We therefore propose a \emph{standard matchertext configuration}
whose alphabet $\Sigma$ is the Unicode/UCS character set~\cite{iso10646ucs},
and whose matcher pairs $\Pi$ consist of
the ASCII parentheses \verb|()|,
square brackets \verb|[]|,
and curly braces \verb|{}|.

The choice of UCS as the character set $\Sigma$
is justified by the fact that machine-readable languages
have largely converged on this standard,
so it is in effect already decided.
In fact, the programming language community has also largely converged
on UTF-8 as the standard way to encode UCS plain text
into flat byte-stream source files --
although encoding is not a primary concern for matchertext
since it operates below the character set abstraction.

\subsubsection{Standardizing on matcher pairs}
\label{sec:design:concrete:standard}

A particular choice of the matcher pairs $\Pi$ is less obvious, however,
and hence demands more careful justification.
We start by ``deferring to authority'':
namely the authority embodied in
the UCS character set we already chose.
The parentheses, open brackets, and curly braces
are the only characters in the ASCII --
or ``\href{https://www.compart.com/en/unicode/block/U+0000}{Basic Latin}'' -- 
code block that are standardized as members of the
\href{https://www.compart.com/en/unicode/category/Ps}{Open Puntuation (Ps)}
and 
\href{https://www.compart.com/en/unicode/category/Pe}{Close Punctuation (Pe)}
character classes.
Exactly as their official names indicate,
these character classes denote characters whose standard purpose
is to serve as open and close punctuation in matched pairs.

\paragraph{Why not the ``angle brackets'' \texttt{<>}?}
Many programming language also use the ASCII characters \verb|<| and \verb|>|
in matching pairs,
such as for generic types in C++ and Java,
or markup in SGML-derived languages such as HTML and XML.
These characters are not standardized as open/close punctuation, however,
but as mathematical less-than and greater-than inequality symbols.
Further, they are used for this purpose in mathematical expressions,
in \emph{unmatched} fashion,
much more pervasively than their occasional use as matchers.
Requiring these characters to be matched in matchertext
would not only conflict with their primary standardized purposes
(\ie would ``defy the authority'' of UCS),
but would make it extremely cumbersome to express
standard mathematical inequalities
(\eg \verb|a < b| in \verb|if| expressions and the like)
in almost all programming languages.
Omitting \verb|<>| from the matched pairs $\Pi$
of the standard matchertext configuration does not
conflict with or prevent their use in pairs in specific languages --
as we will see when we focus on SGML-derived languages later
in \cref{sec:host:ml} and \cref{sec:embed:ml}.
This means only that we do not impose a ``universal'' rule
that they \emph{must} be used \emph{strictly} as matchers,
without exception, throughout all valid matchertext.

\paragraph{Why only the ASCII open/close punctuation?}
The full UCS standard of course includes
much more open and close punctuation.
UCS also includes
\href{https://www.compart.com/en/unicode/category/Pi}{Initial Puntuation (Pi)}
and 
\href{https://www.compart.com/en/unicode/category/Pf}{Final Punctuation (Pf)}
character classes specifically for quotation marks
intended for use in pairs
(\eg «quote» or “quote”).
None of these extended UCS characters are commonly used
in machine-readable language syntax, however --
no doubt in part merely by tradition,
but also for the pragmatic reason that only the ASCII punctuation symbols
are directly typeable on most keyboard layouts.
Moreover, the open/close and initial/final punctuation
in the extended UCS blocks
do not occur strictly in pairs.
For example,
there are three different left double-quote characters
(codes 201C, 201E, and 201F)
that potentially match with the right double-quote character (code 201D),
depending on linguistic culture and typographical style.
Thus, deciding \emph{which} character pairs should or should not match
would become much more complex.
While specific languages are free to use any UCS punctuation
for their own language-specific structural or stylistic purposes,
it seems simplest and safest to restrict the matchertext set $\Pi$ of
\emph{strictly-matching} pairs to the ASCII matchers alone.

\paragraph{Why all three ASCII matcher pairs?}

We could of course be even more selective
in choosing the set of matcher pairs $\Pi$.
We could take only one matcher pair, for example:
either parentheses \emph{or} square brackets \emph{or} curly braces.
However, all of these matcher pairs are used quite pervasively,
in a variety of different structural roles in different languages,
and it is not readily apparent what principle would justify
choosing one of these matcher pairs over the others
to play a distinguished, globally-enforced matching role in matchertext.
Moreover, interspersing multiple distinct matcher pairs in structured text
in practice provides useful redundancy
that helps detect errors more quickly and localize them more precisely.
For example, it is much clearer where the missing close bracket is
in the string \verb|[{}([){}]|
than in the similar but more homogeneous string \verb|[[][[][]]|.
Finally,
any host language must use \emph{some} matcher pair
to delimit embedded matchertext strings,
as discussed above in \cref{sec:design:abstract:embed}.
Including all three ASCII matcher pairs in $\Pi$
thus gives languages maximum syntactic freedom
in defining the syntax of matchertext embeddings
(\ie a choice of three matcher pairs rather than a single prescribed pair).


\subsection{Matchertext configuration variations}
\label{sec:design:concrete:variations}

Even if well-justified,
we cannot expect the standard matchertext configuration
as defined above to be a perfect or painless fit
for all situations in which string embedding is useful.
Including any matcher pair in $\Pi$ has the cost
of requiring those matchers to be escaped
in string literals and comments, for example,
as we detail later in \cref{sec:embed}.
There may be legitimate or even unavoidable reasons
to use other matchertext configurations in some cases,
keeping in mind that doing so reduces interlanguage embedding compatibility.

In general, deviations from the standard matchertext configuration
can be either \emph{tightening} (more restrictive),
\emph{loosening} (less restrictive), or a combination.

\subsubsection{Tightening variations}
\label{sec:design:concrete:variations:tight}

A matchertext configuration $(\Sigma',\Pi')$
is a \emph{tightening} of
the standard matchertext configuration $(\Sigma,\Pi)$ defined earlier
if it only removes characters from the alphabet ($\Sigma' \subseteq \Sigma$)
and/or makes additional matcher pairs sensitive ($\Pi' \supseteq \Pi$).
A string in the tightened matchertext configuration
may be copied verbatim to an embedding context
expecting the standard matchertext configuration,
but not necessarily in the other direction.

As we detail later in \cref{sec:host:uri},
uniform resource identifiers (URIs)~\cite{rfc3986} traditionally allow
only graphical characters -- and no spaces or control codes for example --
in order to make them manually transcribable.
To serve this transcribability purpose,
significant spaces and control codes must not appear \emph{anywhere} in a URI,
even in an embedded matchertext substring.
Thus, URIs may represent a justifiable use-case
for an alternate matchertext configuration that removes
the non-graphical characters from the alphabet.
This would unfortunately mean that a string cannot, in general,
be copied from a standard matchertext language into a matchertext URI
without transformation (\ie escaping spaces and control codes).

\subsubsection{Loosening variations}

A matchertext configuration $(\Sigma',\Pi')$
is a \emph{loosening} of
the standard matchertext configuration $(\Sigma,\Pi)$
if it only adds characters to the alphabet ($\Sigma' \supseteq \Sigma$)
and/or removes sensitive matcher pairs ($\Pi' \subseteq \Pi$).
A string in the standard matchertext configuration
may be copied verbatim to an embedding context
supporting the loosened matchertext configuration,
but not necessarily in the other direction.

A loosened matchertext configuration might be justified,
for example, if it is deemed critical to embed strings in some language $L_e$
that frequently makes unmatched uses of some ASCII matchers,
and the pain of escaping or otherwise adapting that syntax is deemed too great.

Mathematical notation, for example,
sometimes uses
``mismatched'' parentheses and square brackets
to represent half-open/half-closed intervals.
That is,
$[0,1)$ typically means any real number $r$ greater than or equal to zero
but strictly less than one ($0 \le r < 1$).
A machine-readable language making frequent use of this mathematical notation
might be considered too painful to embed
in a standard matchertext configuration,
and therefore might ``demand'' a looser configuration
in which perhaps only the curly braces are sensitive as matcher pairs.

This example seems fairly hypothetical, however,
as extremely few machine-readable languages
appear to support this mathematical
half-open/half-closed interval notation anyway.
Languages that do support some form of half-open/half-closed syntax
often do so with other, more matchertext-friendly notation.
Swift~\cite{apple22swift}, for example,
supports \href{https://docs.swift.org/swift-book/LanguageGuide/BasicOperators.html#ID73}{\emph{half-open range} syntax} like \verb|1..<4|
for the sequence of integers starting from and including 1,
up to but not including 4.
This syntax is perfectly compatible with the standard matchertext configuration
because it uses the mathematical inequality operators,
rather than unmatched matchers,
to express the range's open upper endpoint.

