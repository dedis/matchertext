\section{Background: needs and pitfalls of interlanguage embedding}
\label{sec:bg}

The practice of
embedding strings from one language into another
is ubiquitous --
as is the pain of having to ``escape'' embedded strings
to protect them from misinterpretation by the host language processor.
This section briefly explores a few of these common existing practices
and the syntactic composition problems they create.

\subsection{Special-purpose languages}

Many special-purpose language syntaxes exist
almost solely for embedded use in other syntactic contexts.
A few particularly common examples of such ``little languages''
include regular expressions (REs),
uniform resource identifiers (URIs),
JavaScript Object Notation (JSON),
and Structured Query Language (SQL).

Perhaps the most classic ``little language'' is regular expression (RE) syntax,
commonly used for pattern-based searches and replacements in freeform text.
RE syntax traditionally uses many punctuation characters for special purposes,
but must also allow arbitrary embedded text to be matched literally.
RE syntax therefore makes heavy use of ``backslash-escaping'' in literal text:
\eg the RE \verb|.*|
matches any number of arbitrary characters other than newlines,
while one must write \verb|\.\*| to match the literal string `\verb|.*|'.
Because REs themselves are often embedded in another language --
frequently a C-inspired language that \emph{also} uses backslash-escaping
in string literals --
we must further backslash-escape the backslashes in RE syntax.
A C string literal containing the latter RE above
is written \verb|"\\.\\*"|,
for example.
To match the double-backslash \verb|\\| that begins
a \href{https://en.wikipedia.org/wiki/Path_(computing)#Universal_Naming_Convention}{UNC name} (as in \verb|\\host\path\file|),
the backslashes must be doubled to become \verb|\\\\| as an RE,
then doubled again to become \verb|"\\\\\\\\"| as a C sring literal
containing that RE.
This confusing multi-level explosion of backslash escapes
has been aptly dubbed
\href{https://en.wikipedia.org/wiki/Leaning_toothpick_syndrome}{\emph{leaning toothpick syndrome}}.

\subsection{The complexity of multi-level escaping}

As \cref{fig:territorial-integrity} illustrates,
each additional level of embedding in traditional syntax
adds a set of escaping requirements that the writer (or reader) of code
must carefully consider and apply correctly --
\emph{in the correct order} --
in order to ``guide'' the embedded text
through the levels of host syntax that the text is embedded in.
REs and C string literals each require
a \emph{different} set of punctuation characters to be backslash-escaped,
further increasing the cognitive load when embedding.
\href{https://www.pcre.org/original/doc/html/pcrepattern.html}{PCRE syntax}
helpfully promises that a backslash followed by any non-letter
always ``takes away any special meaning that character may have'',
so one may fall back on just \emph{escaping all punctuation}
instead of remembering which characters must actually be escaped.
But this practice feels like a band-aid at best,
and yields even more ``leaning toothpicks.''

The above examples also illustrate how each level of embedding
can multiply the length of embedded strings by a factor of 2 or more,
yielding in the worst case
an exponential string-length explosion with the number of embedding levels.
While more than two levels of embedding may not be that common,
they do occur.

\xxx{
Convenience challenges.
Error-proneness issues.
Destructive interaction of inner and outer escaping mechanisms.

Shells, scripts, and command-line escaping.
Log files.

Larger languages.  JavaScriopt/TypeScript within HTML/XHTML.
Template-driven authoring languages (e.g., Hugo).
}

\subsection{A proliferation of quoting conventions}

Escaping issues such as those above have in part led many languages
to support multiple different types of quotes with different escaping rules.
Many languages allow string literals
to be either single-quoted or double-quoted,
so that quote characters of one type can be used literally
within string literals of the other,
as in \verb|"'"| or \verb|'"'|.

Some languages disable escape sequences in one form of string literal
so that backslashes may appear literally without multiplying in number:
\eg single-quoted Bourne shell strings (\verb|'\'| is the same as \verb|"\\"|)
or backtick-quoted raw string literals
in Go (\verb|`\`| is like \verb|"\\"|).
But without escapes it becomes more difficult to include
the forbidden terminating quote literally in the string:
typically one must compose multiple strings,
like \verb|"it's "+'"quoted"'|.
Some languages such as Python allow triple-quoted multiline strings
to make it less likely that the terminating sequence is needed in the literal:
\eg \verb|'''|$\dots$\verb|'''| or \verb|"""|$\dots$\verb|"""|.
But such a sequence may still need to appear, of course --
especially in written examples of exactly this syntax for example.

Some languages further mitigate this problem
by offering an effectively-unlimited number of delimiter pairs.
\href{https://docs.swift.org/swift-book/LanguageGuide/StringsAndCharacters.html#ID286}{Extended string literals in Swift}, for example,
surround a quoted sequence with a balanced number of \verb|#| signs:
\eg \verb|#"|$\dots$\verb|"#|, \verb|##"|$\dots$\verb|"##|, etc.
\href{https://www.lua.org/manual/5.1/manual.html}{Lua}
similarly offers \emph{long bracket} quotations
like \verb|[=[|$\dots$\verb|]=]|, \verb|[==[|$\dots$\verb|]==]|, etc.
This approach has the appeal that for any string to be delimited,
there always \emph{exists} some delimiter pair
that can quote it unambiguously.
But the delimiters must still be carefully matched to the quoted string,
or vice versa.
It is still not possible to embed \emph{any} string of a broad class
verbatim into \emph{any} ``hole'' or template in a host language
without thinking about, and potentially adapting,
either the choice of delimiters or the embedded string.
Further, the worst-case ``cost'' of embedding in terms of string expansion
still increases with each level of nesting --
in this case
at least only linearly, rather than exponentially, in the number of levels.
We would prefer, however, if embedding
required \emph{no} expansion with increased depth.


\xxx{ explore: PHP-generated HTML with inline JavaScript.
See for example \href{http://www.zedwood.com/article/how-to-properly-escape-inline-javascript}{How to properly escape inline javascript} }

\subsection{When security goes wrong}

The syntactic complexity of correctly embedding strings into code
has created several broad classes of security vulnerabilities,
where untrusted (typically user-entered) strings intended to be embedded
can maliciously ``trick'' an application into interpreting parts of the string
in the host language context.

SQL injection attacks~\cite{clarke12sql}, for example,
typically arise from the common practice
of embedding user-entered strings into string literals
within SQL query templates.
If a server composes an SQL query with a clause like
\verb|"WHERE name='"+userName+"'"|, 
and the untrusted \verb|userName| can be maliciously crafted
to contain an unescaped single quote,
then the attacker can prematurely terminate the SQL string literal
and add other SQL clauses, like \verb|OR '1'='1'|
to make the clause unconditionally true regardless of \verb|userName|.

Cross-site scripting (XSS) attacks~\cite{fogie07xss}
similarly exploit errors in the ubiquitous practice
of embedding content from an untrusted source --
such as fields from an HTML form --
into HTML or other markup without proper ``sanitization'' via escaping.
Suppose, for example, that one user of a Web-based discussion forum
can post a message containing an HTML \verb|<script>| tag --
which the discussion server then inserts into corresponding pages
viewed by \emph{other} users of the forum.
The injected JavaScript code can then potentially steal
authentication cookies or other private information
from \emph{all} users on the site
who might unwittingly read the maliciously-crafted message.

These and other broad classes of syntactic confusion attacks
have led to the security-critical practice
of sanitizing all potentially-untrusted user input
before embedding it into security-sensitive code of any kind --
whether SQL queries, HTML markup, or other host languages.
The forms of sanitization needed in a particular context
unfortunately tend to be complex and intricately dependent on
the syntax and semantics of the host language
that the untrusted content is to be embedded in.
Version updates in the host language or associated libraries,
which the application might not always track immediately,
can easily introduce new syntactic attack vectors
that the application
has not yet countered with appropriate sanitization logic.

We do not expect any syntactic discipline, including matchertext,
to eliminate the need to sanitize untrusted inputs.
If a future SQL query or Web form is designed to accept
embedded matchertext from an untrusted source, for example,
then it will likely still be security-critical to check
that the untrusted content \emph{is indeed valid matchertext},
and reject it if not.
However, a passive \emph{verification} like this
can be much simpler, and hence less bug-prone,
than a content-modifying \emph{transformation},
to escape all characters or sequences that might be ``sensitive''
in the host language.
Further, this security-critical check
could also be more uniform across host languages --
\ie checking only that the three ASCII matcher pairs are matched corrrectly
throughout the untrusted content,
rather than deeply verifying and/or transforming
based on the complex syntax of a particular host language.
Thus, while matchertext will not eliminate the need for sanitization,
it might tighten and simplify
the function of the most security-critical ``checkpoint'' --
namely checking that embedding content from an untrusted source
preserves the structural integrity
of the host language it is embedded in.

