\section{Background: needs and pitfalls of interlanguage embedding}
\label{sec:bg}

\subsection{Needs for embedding}

special-purpose ``little languages'' that are usually embedded.
URIs. regexps. JSON. SQL.

Shells, scripts, and command-line escaping.
Log files.

Larger languages.  JavaScriopt/TypeScript within HTML/XHTML.
Template-driven authoring languages (e.g., Hugo).

\xxx{ explore: PHP-generated HTML with inline JavaScript.
See for example \href{http://www.zedwood.com/article/how-to-properly-escape-inline-javascript}{How to properly escape inline javascript} }

\href{https://en.wikipedia.org/wiki/Leaning_toothpick_syndrome}{"leaning toothpick syndrome"}

\subsection{What often goes wrong}

Convenience challenges.

Error-proneness issues.

Destructive interaction of inner and outer escaping mechanisms.

Multiple levels of embedding: exploding complexity of manual escaping;
potentially-exponential expansions with automatic escaping.
(e.g., C-like escaping: `\verb|\|' becomes `\verb|\\|'
and `\verb|"|' becomes `\verb|\"|',
so $2^l$ size with $l$ levels of escaping.)


Long quotes.  
triple-quotatins might indeed be less likely to appear
in a big text cut-and-pasted into a file,
but it still \emph{may} unexpectely appear.
The same goes with long-inclusion ...

\subsection{When security goes wrong}

armoring of untrusted inputs,
and what happens when that isn't done [correctly].

TODO: Review the security literature for this kind of security bug.

